\documentclass[11pt, oneside]{article}   	% use "amsart" instead of "article" for AMSLaTeX format
\usepackage[margin=0.25in,includefoot]{geometry}                		% See geometry.pdf to learn the layout options. There are lots.
\geometry{letterpaper}                   		% ... or a4paper or a5paper or ... 
%\geometry{landscape}                		% Activate for rotated page geometry
%\usepackage[parfill]{parskip}    		% Activate to begin paragraphs with an empty line rather than an indent
\usepackage{graphicx}				% Use pdf, png, jpg, or eps§ with pdflatex; use eps in DVI mode
								% TeX will automatically convert eps --> pdf in pdflatex		
\usepackage{amssymb}
%SetFonts

%SetFonts


\title{Descriptive Analysis}
\author{Rachel Anderson}
\date{\today}							% Activate to display a given date or no date

\begin{document}
\maketitle
%\section{}
%\subsection{}

\section{Safe Harbor Rules}

Begin-construction requirement enacted in early 2013 for PTC and ITC that was since clarified.  Modified to mimic language in 1603 grant program and bonus depreciation, bond financing, etc. 

2013 guidance:  facility placed in service within two calendar years measured from statuory deadline to begin construction

\subsection{Solar}
\begin{itemize}
\item Solar ITC safe harbor rule made on February 9, 2018, the Bipartisan Budget Act of 2018.   
\item Original ITC deadline was placed in service date of December 31, 2015. Then ITC got extended on December 18, 2015 for projects placed in service before January 1, 2024.  Even though credit rates awarded will depend on begin construction date after 2018, all projects must follow placed in service deadline.  
\end{itemize}
\subsection{Wind}
\begin{itemize}
\item Safe Harbor for PTC (wind) more complicated:
\end{itemize}



\section{Case Study 1: California}

\subsection{RPS Program}
\begin{itemize}
\item 2002:  Program established by Senate Bill (SB) 1078 with initial requirement that 20\% of retail sales must be served by renewable resources by 2017.  
\item 2003: Energy Action Plan I accelerates 20\% deadline to 2010
\item 2006: acceleration codified into law in SB 107
\item 2004 Energy Action Plan II examined further goal of 33\% by 2020
\item 2005: Assembly Bill 200 modified requirements for electric corporations that serve customers outside of California and have <60k customer accounts in CA 
\item 2006: Executive Order establishes targets to increase bioenergy use/production
\item 2006: Assembly Bill 1969 requires electrical corporations to purchase, at CPUC set price, renewable energy output from public water and wastewater facilities up to 1 MW
\item 2007: modifies supplemental energy payments process (?)
\item 2008: AB 3048 makes minor technical changes to clarify code(?)
\item 2008: SB 280 amends PU Code 399.20 to make feed-in tariff for all renewable generators (previously limited to water/wastewater facilities) and increase program cap to 500 MW (from 250 MW) 
\item 2015: SB 350 mandated 50\% RPS by 2030, with interim annual targets with three-year compliance periods; requires 65\% of RPS procurement to be derived from long-term contracts of 10+ years
\item 2017: All electricity retail sellers had an interim target to serve at least 27\% of their load with RPS-eligible resources by December 31, 2017
\item 2018: SB 100 increases RPS to 60\% by 2030, requires all state's electricity to come from carbon-free sources by 2045
\end{itemize}

Administered by CPUC for California's retail sellers of electricity, which include large and small IOUs, electric service provides, and community choice aggregators.  California Energy Commission certifies generation facilities as eligible resources and enforcing requirements for POUs.  

The CPUC determines annual procurement targets and enforces compliance, reviews IOU procurement plans, reviews contracts for RPS-eligible energy, establishes standard terms and conditions used by IOUs in their contracts for eligible sources.  Renewable energy tracking system is maintained by WREGIS.  

In general, retail sellers met or exceed the interim 27\% target and are on track to achieve their compliance requirements. 

The big three IOUs collectively served 36\% of sales with renewables.  Small and Multi-jurisdictional utilities and ESPs had roughly 27\%; CCAs served 50\% with renewable power.  

RPS resources include wind, solar PV, solar thermal, hydro, geothermal, bioenergy.  
\begin{figure}
\caption{Investor-owned utility RPS Contracts}
\includegraphics[width=\textwidth]{../../Figures/iou_rps_map.png}
\end{figure}

\subsection{RPS Procurement Programs}
\begin{enumerate}
\item Utility Scale Request for Offer
\begin{itemize}
\item The three largest investor-owned utilities in California contract for long-term, utility-scale supplies of renewable energy through annual competitive requests for offers called “RFOs.” The CPUC established this program for utility procurement of renewable energy to ensure that the utilities meet their RPS obligations through a transparent process.
\end{itemize}
\item Renewable Auction Mechanism
\begin{itemize}
\item The Renewable Auction Mechanism, or RAM, is a simplified market-based procurement mechanism for renewable distributed generation (DG) projects greater than 3 MW and up to 20 MW on the system side of the meter.
\end{itemize}
\item RPS Feed-in Tariff Program 
\begin{itemize}
\itema feed-in tariff program for renewable generators less than 3 MW in size. This program offers up to 493.6 MW to eligible projects through a fixed-price standard contract to export electricity to California’s three large investor owned utilities (IOUs). The ReMAT Program replaced the AB 1969 Feed-in Tariff Program in 2013.
\end{itemize}
\item Utility Solar Rooftop
\begin{itemize}
\item The Commission authorized Southern California Edison (SCE), Pacific Gas and Electric Company (PG\&E), and San Diego Gas \& Electric Company (SDG\&E) to own and operate solar PV facilities (UOG) as well as to execute solar PV power purchase agreements with independent power producers (IPP) through a competitive solicitation process.
\end{itemize}
\end{enumerate}


\subsection{Regulation}
CPUC regulates three IOUs (private electricity and natural gas providers) that comprise approximately three-quarters of electricity supply in California:  
\begin{enumerate}
\item Pacific Gas and Electric
\item San Diego Gas and Electric
\item Southern California Edison 
\end{enumerate}

IOUs purchase power through contracts and operate their own generation facilities.  Resumed procurement in 2002 after energy crisis; every 2 years CPUC holds Long Term Procurement Plan proceeding to review and adopt the IOUs' ten-year procurement plans; establishes rules for rate recovery of procurement transactions.  

Utility rates are set to recover costs and earn a reasonable return as profits for investors in return for risk they bear for investing in new facilities.  

Rest of California is powered by publicly owned utilities.  Largest is Los Angeles Department of Water and Power, which provides service to 3.9 million customers.  By contrast, these are non-profit public entities managed by locally elected officials or public employees (and would need to partner with tax equity investors....).  Public utilities have access to tax-free bonds and co-ops have access to low-interest loans usually at the local level. POUs are not regulated by the CPUC and are not subject to the same energy efficiency mandates as the IOUs; however Energy Commission evaluates POUs.    


Relevant Policies
\begin{itemize}
\item 2010: California Public Utilities Commission adopts Renewable Auction Mechanism
\begin{itemize}
\item Program to help IOUs procure RPS eligible generation in streamlined fashion by providing standard non-negotiable contracts and a standardized valuation process.  Projects can then be submitted to the CPUC through an expedited regulatory review process.
\item Initial program for renewable DG projects between 3-20 MW, and 1,000 MW...?  No longer has caps, no longer limits geographic location of projects to service territory of the IOUs. 
\end{itemize}
\end{itemize}



Facilitates quick simple transactions where projects can come online relatively sooner; IOUs submit annual RPS procurement plan filings  

California Solar Initiative:  pays for energy produced by solar energy systems on monthly basis over 5-year period of time at fixed incentive rate (\$/kWh) 

Performance-based incentive for solar motivates installers and owners to focus on proper installation, maintenance, and performance of systems, since the payment is based upon actual energy produced.  This provides policy makers and regulators assurances that the incentives are not squandered on systems with poor performance. 

\section{Case Study 2: Texas} 

RPS program requires 10,000 MW capacity by 2020...? 

Compliance Premium awarded by ERCOT in conjunction with a REC that is generated by non-wind renewable generation.

 
Number Of Installations: 63,466

``Texas is poised to become a nationwide leader in solar energy, with more than 4 GW of capacity expected to be installed over the next 5 years, with appropriate state policy that removes market barriers and recognizes solar's benefits."

\section{Things to look up}
\begin{itemize}
\item When did placed in service switch to start construction? 
\end{itemize}

\section{Pre-2005 Credits}
\begin{itemize}
\item Energy Policy Act of 1992
\begin{itemize}
\item made 10\% ITC for solar and geothermal permanent
\item first enactment of PTC for electricity generated using wind or closed-loop biomass 
\end{itemize}

\item Working Families Tax Relief Act of 2004
\begin{itemize}
\item extended PTC through December 31, 2005; at this point PTC for poultry waste, too
\end{itemize}

\item American Jobs Creation Act of 2004 
\begin{itemize}
\item Added open-loop biomass (including agricultural livestock waste), geothermal, solar, small irrigation power, and municipal solid waste (landfill gas and trash combustion); but limited to five-year PTC period
\item Open-loop biomass, small irrigation power and municipal solid waste received only half credit
\item Introduced PTC for refined coal, with rate of \$4.375 per ton on qualifying serviced placed in service before January 1, 2009
\end{itemize}

\end{itemize}

\section{Relevant Policies 2005--Present}
\begin{itemize}

\item After December 31, 2005, PTC for solar expires 

\item Energy Policy Act of 2005 (8/8/05)
\begin{itemize}
\item increased solar ITC from 10\% to 30\% for 2006 and 2007
\item 30\% ITC for fuel cell power plants, 
\item 10\% for stationary microturbine power plants placed in service during 2006 and 2007
\item extended PTC for all facilities except solar energy and refined coal for two years, through 2007
\item added PTC for hydropower (half-credit) and Indian Coal (seven-year period, \$1.50 per ton for first four years, then \$2.00 per ton for last three years)
\item extended PTC period from 5 to 10 years for all qualifying facilities (other than Indian Coal) for all qualifying facilities placed in service after August 8, 2005
\end{itemize}

\item Tax Relief and Health Care Act of 2006 (12/20/06)
\begin{itemize}
\item extended ITC through 2008
\item extended PTC through 2008 for all technologies but solar, refined coal and Indian coal through 2008
\end{itemize}

\item Emergency Economic Stabilization Act of 2008 (10/3/08)
\begin{itemize}
\item extended credits for solar, fuel cells and microturbines through December 31, 2016
\item provided 10\% credit for geothermal heat pump property
\item 30\% credit for qualified small wind energy property
\item 10\% credit for combined heat and power (CHP)
\item ITC all with placed-in-service deadline of December 31, 2016
\item PTC for wind and refined coal extended through 2009
\item PTC for closed-loop and open-loop biomass, geothermal, small irrigation, municipal solid waste, and hydropower extended two years through 2010
\item added PTC for marine and hydrokinetic renewable energy; and new credit for steel industry fuel
\end{itemize}

\item American Recovery and Reinvestment Act of 2009 (2/17/09)
\begin{itemize}
\item extended PTC for wind through 2012 and for other technologies through 2013
\item allowed ITC or one-time grant in lieu of PTC for property placed in service or start construction in 2009 and 2010
\end{itemize}

\item Tax Relief, Unemployment Insurance Reauthorization, and Job Creation Act of 2010
\begin{itemize} 
\item extended grant program for one year (through 2011)
\end{itemize}

\item American Taxpayer Relief Act of 2012 (1/2/13)
\begin{itemize}
\item extended PTC for wind through 2013, 
\item changed placed-in-service PTC requirement to start construction 
\end{itemize}

\item Tax Increase Prevention Act of 2014 (12/19/14)
\begin{itemize}
\item PTC and ITC in lieu of PTC option retroactively extended through 2014
\end{itemize}

\item Consolidated Appropriations Act, 2016 (12/18/15)
\begin{itemize}
\item extended the 30\% credit rate for solar electric or heating property (but not fiber-optic) through 2019
\item Termination date changed from placed-in-service deadline to construction start date
\item Credit set at 26\% for construction beginning in 2020; 22\% for 2021
\item To qualify for a rate in excess of 10\%, property must be placed in service by December 31, 2023
\item extended PTC expiration date for nonwind facilities through end of 2016
\item extended ITC in lieu of PTC option through 2016
\item extended PTC for Indian Coal through 2016
\item removed placed in service limit for Indian Coal
\item extended PTC for wind through 2019 with reduced rates each year
\item A permanent 10\% ITC will remain for solar and geothermal
\end{itemize}

\item Bipartisan Budget Act of 2018 (2/9/18)
\begin{itemize}
\item retroactively extended PTC for nonwind and Indian Coal for tax year 2017
\item extended ITC for five years for fiber-optic solar, fuels cells, small wind, microturbine, CHP, geothermal heat pump property
\item For property eligible for 30\% credit rate through 2019, credit rate reduced according to solar reduction schedule
\item All termination dates changed to construction start deadlines
\end{itemize}
\end{itemize}

\section{EIA Annual Capacity Data}


\includegraphics[width=0.9\textwidth]{../../Figures/tot_renew_cap.png}


\subsection*{Comments}
\begin{itemize}
\item Vertical lines at 2006 (ITC and PTC increase), 2009-2012 (loan grant program), 2016 (initial expiration)
\item Renewables growth driven by solar (after 2012) and wind (after 2006)
\end{itemize}

\begin{figure}[h!]
\caption{Renewable capacity (as stock and percent of total capacity) by region in 2018}
\includegraphics[width=0.4\textwidth]{../../Figures/region_renew_cap.png}
\includegraphics[width=0.4\textwidth]{../../Figures/region_prenew.png}\\
\end{figure}

\begin{figure}[ht]
\caption{FERC Power Markets}
\includegraphics[width=\textwidth]{../../Figures/ferc-markets.png}
\end{figure}

\includegraphics[width=0.85\textwidth]{../../Figures/tot_renew_by_region.png}\\
\includegraphics[width=0.9\textwidth]{../../Figures/p_renew_regions.png}

\begin{figure}
\caption{Renewable Energy Sources by Region}
\includegraphics[width=0.8\textwidth]{../../Figures/renew_comp_regions.png}
\end{figure}
\newpage
\includegraphics[width=0.9\textwidth]{../../Figures/tot_renew_by_region_tech.png}

\newpage

\includegraphics[width=0.9\textwidth]{../../Figures/relative_growth_renew_tot.png}
\\

Vertical lines indicate the following policy changes:
\begin{itemize}
\item  2006:  First year of 30\% ITC, 10-year PTC
\item 2009-2012: Loan grants in lieu of ITC; ITC or loan grant in lieu of PTC
\item  2016: Original end of ITC  
 \end{itemize}
There appear to be spikes in renewable capacity additions that align with these programs. 

\includegraphics[width=0.85\textwidth]{../../Figures/growth_rates.png}\\
\includegraphics[width=0.85\textwidth]{../../Figures/yearly_changes.png}



\newpage
\includegraphics[width=0.85\textwidth]{../../Figures/after_2005_stock.png}\\
\includegraphics[width=0.85\textwidth]{../../Figures/after_2005_additions.png}

\newpage

\section{To Do with capacity data}
\begin{enumerate}
\item Renewable capacity in regulated vs. non-regulated... (I started but need to weight appropriately)
\item Find out details about placed in service requirements for tax credits
\item Calculate variable = tax credit rate for each technology + RPS variables
\end{enumerate}

\newpage
\section{Proposed generator data}
\subsection{Wind and solar projects only}
\begin{figure}[ht]
\caption{this is a caption}
\includegraphics[height=0.5\textheight]{../../Figures/ProposedGenFigs/wind_solar_ngen.png}
\includegraphics[height=0.5\textheight]{../../Figures/ProposedGenFigs/wind_solar_cap.png}
\end{figure}

\begin{figure}
\caption{this is a caption}
\includegraphics[height=0.5\textheight]{../../Figures/ProposedGenFigs/year_to_year.png}
\end{figure}




\end{document}  