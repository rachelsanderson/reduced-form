\documentclass[12pt, oneside]{article}   	% use "amsart" instead of "article" for AMSLaTeX format
\usepackage[margin=0.5in,includefoot]{geometry}                		% See geometry.pdf to learn the layout options. There are lots.
\geometry{letterpaper}                   		% ... or a4paper or a5paper or ... 
%\geometry{landscape}                		% Activate for rotated page geometry
%\usepackage[parfill]{parskip}    		% Activate to begin paragraphs with an empty line rather than an indent
\usepackage{graphicx}				% Use pdf, png, jpg, or eps§ with pdflatex; use eps in DVI mode
								% TeX will automatically convert eps --> pdf in pdflatex		
\usepackage{amssymb}
\usepackage{booktabs}
\usepackage{multirow}
\usepackage{setspace}

%SetFonts

%SetFonts


\title{Proposed Generator Data Analysis}
\author{Rachel Anderson}
\date{\today}							% Activate to display a given date or no date

\begin{document}
\maketitle

\section{Data description}
\doublespacing
The data are intended to constitute a complete inventory of electric generating units located at facilities
with a minimum on-site nameplate capacity of one megawatt (MW). The data collected include:
\begin{itemize}
\item The location (state, county, balancing authority, latitude and longitude) of a power plant
\item The ownership of generating units (including designations of joint ownership)
\item The capacity and energy source used by each generating unit at the plant
\item The status of the plant as of December 31 of the reporting year (e.g., operational, standby, or
retired)
\item For steam electric plants, individual boiler characteristics, cooling-water systems, and emission
control systems in non-nuclear plants (see a description in sub section Power Plant
Environmental Controls and Estimated Emissions)
\end{itemize}
In addition to collecting data on existing generating units, EIA also collects data on \textbf{proposed plants and
plants under construction.} To be included, the plant must be scheduled for commercial operation
within 10 years for coal and nuclear plants and within 5 years for all other types of plants.
\\
\hspace{10pt}
\\
The analysis below uses data on the proposed plants. Each observation consists of a single generating unit.  I use the first year that each generating unit appears in my data, and so ``scheduled completion year" refers to the initial scheduled completion date.  Later on, I will look at how frequently projects are delayed or cancelled, by linking the proposed generator data with the operating generator data. 

\pagebreak
\section{Credit eligibility by year}

\begin{table}[h!]
\caption{Solar ITC Elgibility}
\vspace{10pt}
\begin{tabular}{lp{5cm}p{7cm}}
\toprule
Loan grant eligible & Begin construction between 2009-2011 & In my data, determine by first year appearing and ``status"\\
\midrule
ITC Round 1 (2008-2016) & Placed in service by 12/31/2016 & In my data, scheduled completion year $\leq$ 2016 \\
\midrule
ITC Round 2 (2017-2023) & Credit rate determined by begin construction, but eligibility requires placed in service by 12/31/2023 & In my data, requires looking at initial proposed year and scheduled completion year \\
\bottomrule
\end{tabular} 
\end{table}

However, I will meet with a librarian to fill a table like this: 


\begin{table}[h]
\caption{Federal Tax Credit Eligibility}
\begin{tabular}{p{4cm}p{1.5cm}cp{4cm}p{3.5cm}l}
\toprule
Legislation                                  & Date & Credit   & Eligible resources                                                  & Begin construction & Placed in service \\
\midrule
Emergency Economic  & 10/3/08          & 30\% ITC & Solar                                                               &                                & 1-Jan-17                      \\Stabilization Act of 2008
                                             &                  & 30\% ITC & Fuel cell, microturbine                                             &                                & 31-Dec-16                     \\
                                             &                  & 10\% ITC & Geothermal heat pump                                                &                                &                               \\
                                             &                  & 10\% ITC & ``Efficient" combined heat and power systems with 15-50 MW capacity &                                &                  \\
                                             \bottomrule            
\end{tabular}
\end{table}

\section{Summary of Analysis}
\singlespacing
\begin{itemize}
\item I examine the relationship between total capacity and number of projects by proposal and scheduled scheduled completion dates.
\item Proposal date refers to the first year a unit appears in the data; it proxies for ``begin construction".
\item Scheduled construction date proxies for intent to comply with placed-in-service deadlines.
\item I look for different patterns across regulated and deregulated states. 
\end{itemize}

\newpage
\begin{figure}[!hb]
\caption{Size and Number of Projects by ``begin construction" year, all technologies}
\includegraphics[height=0.5\textheight]{../../Figures/ProposedGenFigs/proposed_gen_tot_cap.png}\\
\includegraphics[height=0.5\textheight]{../../Figures/ProposedGenFigs/proposed_gen_nGen.png}
\end{figure}

\begin{figure}[!hb]
\caption{Size and Number of Projects by initial scheduled completion date}
\includegraphics[height=0.5\textheight]{../../Figures/ProposedGenFigs/complete_dates_nGen.png}
\includegraphics[height=0.5\textheight]{../../Figures/ProposedGenFigs/complete_dates_tot_cap.png}
\end{figure}

\begin{figure}[!hb]
\caption{Wind and solar projects by proposal date}
\includegraphics[height=0.5\textheight]{../../Figures/ProposedGenFigs/year1_ws_tot_cap.png}
\includegraphics[height=0.5\textheight]{../../Figures/ProposedGenFigs/year1_ws_nUnits.png}
\end{figure}

\begin{figure}[!hb]
\caption{Wind and solar projects by scheduled completion date}
\includegraphics[height=0.5\textheight]{../../Figures/ProposedGenFigs/wind_solar_cap.png}
\includegraphics[height=0.5\textheight]{../../Figures/ProposedGenFigs/wind_solar_ngen.png}
\end{figure}

\vfill
\clearpage

\section{Wind and Solar} 
\begin{figure}[h!]
\centering
\caption{Capacity over time}
\includegraphics[height=0.45\textheight]{../../Figures/ProposedGenFigs/cap_solar.png}
\includegraphics[height=0.45\textheight]{../../Figures/ProposedGenFigs/cap_wind.png}
\end{figure}

\begin{center}
\includegraphics[width=0.8\textwidth]{../../Figures/ProposedGenFigs/solar_hist.png}
\end{center}

\begin{figure}[!hb]
\centering
\caption{Scheduled completion year by proposal period}
\includegraphics[width=0.8\textwidth]{../../Figures/ProposedGenFigs/year_to_year.png}
\end{figure}

\input{../../Figures/ProposedGenFigs/wind_solar_region_solar.tex}

\input{../../Figures/ProposedGenFigs/wind_solar_region_wind.tex}

\input{../../Figures/ProposedGenFigs/wind_solar_regulated_solar.tex}

\input{../../Figures/ProposedGenFigs/wind_solar_regulated_wind.tex}

\begin{figure}[ht]
\includegraphics[width=0.85\textwidth]{../../Figures/ProposedGenFigs/ann_tot_cap_solar.png}
\includegraphics[width=0.85\textwidth]{../../Figures/ProposedGenFigs/ann_tot_cap_wind.png}
\end{figure}

\begin{figure}[ht]
\includegraphics[width=0.85\textwidth]{../../Figures/ProposedGenFigs/ann_avg_cap_solar.png}
\includegraphics[width=0.85\textwidth]{../../Figures/ProposedGenFigs/ann_avg_cap_wind.png}
\end{figure}

\begin{figure}[ht]
\includegraphics[width=0.85\textwidth]{../../Figures/ProposedGenFigs/ann_nUnits_solar.png}
\includegraphics[width=0.85\textwidth]{../../Figures/ProposedGenFigs/ann_nUnits_wind.png}
\end{figure}


\end{document}  