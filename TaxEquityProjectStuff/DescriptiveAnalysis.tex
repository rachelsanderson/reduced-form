\documentclass[11pt, oneside]{article}   	% use "amsart" instead of "article" for AMSLaTeX format
\usepackage[margin=0.25in,includefoot]{geometry}                		% See geometry.pdf to learn the layout options. There are lots.
\geometry{letterpaper}                   		% ... or a4paper or a5paper or ... 
%\geometry{landscape}                		% Activate for rotated page geometry
%\usepackage[parfill]{parskip}    		% Activate to begin paragraphs with an empty line rather than an indent
\usepackage{graphicx}				% Use pdf, png, jpg, or eps§ with pdflatex; use eps in DVI mode
								% TeX will automatically convert eps --> pdf in pdflatex		
\usepackage{amssymb}
%SetFonts

%SetFonts


\title{Descriptive Analysis}
\author{Rachel Anderson}
\date{\today}							% Activate to display a given date or no date

\begin{document}
\maketitle
%\section{}
%\subsection{}

\section{Things to look up}
\begin{itemize}
\item When did placed in service switch to start construction? 
\end{itemize}

\section{Pre-2005 Credits}
\begin{itemize}
\item Energy Policy Act of 1992
\begin{itemize}
\item made 10\% ITC for solar and geothermal permanent
\item first enactment of PTC for electricity generated using wind or closed-loop biomass 
\end{itemize}

\item Working Families Tax Relief Act of 2004
\begin{itemize}
\item extended PTC through December 31, 2005; at this point PTC for poultry waste, too
\end{itemize}

\item American Jobs Creation Act of 2004 
\begin{itemize}
\item Added open-loop biomass (including agricultural livestock waste), geothermal, solar, small irrigation power, and municipal solid waste (landfill gas and trash combustion); but limited to five-year PTC period
\item Open-loop biomass, small irrigation power and municipal solid waste received only half credit
\item Introduced PTC for refined coal, with rate of \$4.375 per ton on qualifying serviced placed in service before January 1, 2009
\end{itemize}

\end{itemize}

\section{Relevant Policies 2005--Present}
\begin{itemize}

\item After December 31, 2005, PTC for solar expires 

\item Energy Policy Act of 2005 (8/8/05)
\begin{itemize}
\item increased solar ITC from 10\% to 30\% for 2006 and 2007
\item 30\% ITC for fuel cell power plants, 
\item 10\% for stationary microturbine power plants placed in service during 2006 and 2007
\item extended PTC for all facilities except solar energy and refined coal for two years, through 2007
\item added PTC for hydropower (half-credit) and Indian Coal (seven-year period, \$1.50 per ton for first four years, then \$2.00 per ton for last three years)
\item extended PTC period from 5 to 10 years for all qualifying facilities (other than Indian Coal) for all qualifying facilities placed in service after August 8, 2005
\end{itemize}

\item Tax Relief and Health Care Act of 2006 (12/20/06)
\begin{itemize}
\item extended ITC through 2008
\item extended PTC through 2008 for all technologies but solar, refined coal and Indian coal through 2008
\end{itemize}

\item Emergency Economic Stabilization Act of 2008 (10/3/08)
\begin{itemize}
\item extended credits for solar, fuel cells and microturbines through December 31, 2016
\item provided 10\% credit for geothermal heat pump property
\item 30\% credit for qualified small wind energy property
\item 10\% credit for combined heat and power (CHP)
\item ITC all with placed-in-service deadline of December 31, 2016
\item PTC for wind and refined coal extended through 2009
\item PTC for closed-loop and open-loop biomass, geothermal, small irrigation, municipal solid waste, and hydropower extended two years through 2010
\item added PTC for marine and hydrokinetic renewable energy; and new credit for steel industry fuel
\end{itemize}

\item American Recovery and Reinvestment Act of 2009 (2/17/09)
\begin{itemize}
\item extended PTC for wind through 2012 and for other technologies through 2013
\item allowed ITC or one-time grant in lieu of PTC for property placed in service or start construction in 2009 and 2010
\end{itemize}

\item Tax Relief, Unemployment Insurance Reauthorization, and Job Creation Act of 2010
\begin{itemize} 
\item extended grant program for one year (through 2011)
\end{itemize}

\item American Taxpayer Relief Act of 2012 (1/2/13)
\begin{itemize}
\item extended PTC for wind through 2013, 
\item changed placed-in-service PTC requirement to start construction 
\end{itemize}

\item Tax Increase Prevention Act of 2014 (12/19/14)
\begin{itemize}
\item PTC and ITC in lieu of PTC option retroactively extended through 2014
\end{itemize}

\item Consolidated Appropriations Act, 2016 (12/18/15)
\begin{itemize}
\item extended the 30\% credit rate for solar electric or heating property (but not fiber-optic) through 2019
\item Termination date changed from placed-in-service deadline to construction start date
\item Credit set at 26\% for construction beginning in 2020; 22\% for 2021
\item To qualify for a rate in excess of 10\%, property must be placed in service by December 31, 2023
\item extended PTC expiration date for nonwind facilities through end of 2016
\item extended ITC in lieu of PTC option through 2016
\item extended PTC for Indian Coal through 2016
\item removed placed in service limit for Indian Coal
\item extended PTC for wind through 2019 with reduced rates each year
\item A permanent 10\% ITC will remain for solar and geothermal
\end{itemize}

\item Bipartisan Budget Act of 2018 (2/9/18)
\begin{itemize}
\item retroactively extended PTC for nonwind and Indian Coal for tax year 2017
\item extended ITC for five years for fiber-optic solar, fuels cells, small wind, microturbine, CHP, geothermal heat pump property
\item For property eligible for 30\% credit rate through 2019, credit rate reduced according to solar reduction schedule
\item All termination dates changed to construction start deadlines
\end{itemize}
\end{itemize}

\section{EIA Annual Capacity Data}

\includegraphics[width=0.9\textwidth]{../../Figures/tot_renew_cap.png}

\includegraphics[width=0.85\textwidth]{../../Figures/tot_renew_by_region.png}\\
\includegraphics[width=0.9\textwidth]{../../Figures/p_renew_regions.png}

\begin{figure}
\caption{Renewable Energy Sources by Region}
\includegraphics[width=0.8\textwidth]{../../Figures/renew_comp_regions.png}
\end{figure}
\newpage
\includegraphics[width=0.9\textwidth]{../../Figures/tot_renew_by_region_tech.png}

\newpage

\includegraphics[width=0.9\textwidth]{../../Figures/relative_growth_renew_tot.png}
\\

Vertical lines indicate the following policy changes:
\begin{itemize}
\item  2006:  First year of 30\% ITC, 10-year PTC
\item 2009-2012: Loan grants in lieu of ITC; ITC or loan grant in lieu of PTC
\item  2016: Original end of ITC  
 \end{itemize}
There appear to be spikes in renewable capacity additions that align with these programs. 

\includegraphics[width=0.85\textwidth]{../../Figures/growth_rates.png}\\
\includegraphics[width=0.85\textwidth]{../../Figures/yearly_changes.png}



\newpage
\includegraphics[width=0.85\textwidth]{../../Figures/after_2005_stock.png}\\
\includegraphics[width=0.85\textwidth]{../../Figures/after_2005_additions.png}









\end{document}  